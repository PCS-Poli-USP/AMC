\documentclass[a4paper]{article}

\usepackage[brazilian]{babel}
\usepackage[utf8]{inputenc}
\usepackage[T1]{fontenc}
\usepackage{amssymb}
\usepackage{amsmath}
\usepackage{graphicx}
\usepackage{subfig}
\usepackage{adjustbox}

% Estilo PCSAMC com as personalizações
\usepackage{pcsamc}
\usepackage{enumitem}
\usepackage{xhfill}
\usepackage{multicol}
\setlength{\columnsep}{1.5cm}
%\setlength{\columnseprule}{1pt}

\begin{document}
\element{dissertativas}{
    \def\AMCbeginQuestion#1#2{\par #2\hspace*{1em}}
	\noindent{}{\large{}\textbf{[2,6 pontos] Questão 1}}\par
	\begin{questionmult}{Q1}\scoring{v=-1,MAX=2.6}
	\par\noindent{Texto.}
	
	\AMCOpen{lineup=true,dots=false}{
		Para uso do professor:
		\correctchoice[0]{0}\scoring{b=0,m=0}
		\correctchoice[1]{1}\scoring{b=0.1,m=0}
		\correctchoice[2]{2}\scoring{b=0.2,m=0}
		\correctchoice[4]{3}\scoring{b=0.4,m=0}
		\correctchoice[5]{4}\scoring{b=0.5,m=0}
		\correctchoice[10]{5}\scoring{b=1.0,m=0}
		\correctchoice[15]{6}\scoring{b=1.5,m=0}
		\correctchoice[20]{7}\scoring{b=2.0,m=0}
		\correctchoice[26]{8}\scoring{b=2.6,m=0}
	}
	\par
\end{questionmult}



	
    \def\AMCbeginQuestion#1#2{\par #2\hspace*{1em}}
	\noindent{}{\large{}\textbf{[2,6 pontos] Questão 2}}\par
	\begin{enumerate}[label=\alph*]
	\begin{questionmult}{Q2a}\scoring{v=-1,MAX=1.3}
		\item[{\bf (a)}] {\bf [1,3 pontos]} Texto
		
		\AMCOpen{lineup=true,dots=false}{
			Para uso do professor:
			\correctchoice[0]{0}\scoring{b=0,m=0}
			\correctchoice[1]{1}\scoring{b=0.1,m=0}
			\correctchoice[2]{2}\scoring{b=0.2,m=0}
			\correctchoice[4]{3}\scoring{b=0.4,m=0}
			\correctchoice[5]{4}\scoring{b=0.5,m=0}
			\correctchoice[10]{5}\scoring{b=1.0,m=0}
			\correctchoice[13]{6}\scoring{b=1.3,m=0}
		}
	\end{questionmult}
	
	\begin{questionmult}{Q2b}\scoring{v=-1,MAX=1.3}
		\item[{\bf (b)}] {\bf [1,3 pontos]} Texto
		\AMCOpen{lineup=true,dots=false}{
			Para uso do professor:
			\correctchoice[0]{0}\scoring{b=0,m=0}
			\correctchoice[1]{1}\scoring{b=0.1,m=0}
			\correctchoice[2]{2}\scoring{b=0.2,m=0}
			\correctchoice[4]{3}\scoring{b=0.4,m=0}
			\correctchoice[5]{4}\scoring{b=0.5,m=0}
			\correctchoice[10]{5}\scoring{b=1.0,m=0}
			\correctchoice[13]{6}\scoring{b=1.3,m=0}
		}
	\end{questionmult}
\end{enumerate}


}
	
\element{testes}{
	\def\AMCbeginQuestion#1#2{\par\noindent{\bf Teste #1} #2\hspace*{1em}}
	\begin{question}{Teste 1}\scoring{b=0.6,e=0,v=0,m=0}
		Qual a saída deste código?
\begin{choices}
	\correctchoice{2}
	\wrongchoice{3}
	\wrongchoice{4}
	\wrongchoice{5}
	\wrongchoice{6}
\end{choices}

    \end{question}
}

\element{testes}{
	\def\AMCbeginQuestion#1#2{\par\noindent{\bf Teste #1} #2\hspace*{1em}}
	\begin{question}{Teste 2}\scoring{b=0.6,e=0,v=0,m=0}
		Qual a saída deste código?
\begin{choices}
	\correctchoice{2}
	\wrongchoice{3}
	\wrongchoice{4}
	\wrongchoice{5}
	\wrongchoice{6}
\end{choices}

	\end{question}
}

\element{testes}{		
	\def\AMCbeginQuestion#1#2{\par\noindent{\bf Teste #1} #2\hspace*{1em}}
	\begin{question}{Teste 3}\scoring{b=0.6,e=0,v=0,m=0}
		Qual a saída deste código?
\begin{choices}
	\correctchoice{2}
	\wrongchoice{3}
	\wrongchoice{4}
	\wrongchoice{5}
	\wrongchoice{6}
\end{choices}

	\end{question}
}

\element{testes}{
	\def\AMCbeginQuestion#1#2{\par\noindent{\bf Teste #1} #2\hspace*{1em}}
	\begin{question}{Teste 4}\scoring{b=0.6,e=0,v=0,m=0}
		Qual a saída deste código?
\begin{choices}
	\correctchoice{2}
	\wrongchoice{3}
	\wrongchoice{4}
	\wrongchoice{5}
	\wrongchoice{6}
\end{choices}

	\end{question}
}

\element{testes}{
	\def\AMCbeginQuestion#1#2{\par\noindent{\bf Teste #1} #2\hspace*{1em}}
	\begin{question}{Teste 5}\scoring{b=0.6,e=0,v=0,m=0}
		Qual a saída deste código?
\begin{choices}
	\correctchoice{2}
	\wrongchoice{3}
	\wrongchoice{4}
	\wrongchoice{5}
	\wrongchoice{6}
\end{choices}

	\end{question}
}

\element{testes}{
	\def\AMCbeginQuestion#1#2{\par\noindent{\bf Teste #1} #2\hspace*{1em}}
	\begin{question}{Teste 6}\scoring{b=0.6,e=0,v=0,m=0}
		Qual a saída deste código?
\begin{choices}
	\correctchoice{2}
	\wrongchoice{3}
	\wrongchoice{4}
	\wrongchoice{5}
	\wrongchoice{6}
\end{choices}

	\end{question}
}

\element{testes}{
	\def\AMCbeginQuestion#1#2{\par\noindent{\bf Teste #1} #2\hspace*{1em}}
	\begin{question}{Teste 7}\scoring{b=0.6,e=0,v=0,m=0}
		Qual a saída deste código?
\begin{choices}
	\correctchoice{2}
	\wrongchoice{3}
	\wrongchoice{4}
	\wrongchoice{5}
	\wrongchoice{6}
\end{choices}

	\end{question}
}

\element{testes}{
	\def\AMCbeginQuestion#1#2{\par\noindent{\bf Teste #1} #2\hspace*{1em}}
	\begin{question}{Teste 8}\scoring{b=0.6,e=0,v=0,m=0}
		Qual a saída deste código?
\begin{choices}
	\correctchoice{2}
	\wrongchoice{3}
	\wrongchoice{4}
	\wrongchoice{5}
	\wrongchoice{6}
\end{choices}

	\end{question}
 }

  % Início da impressão de cada prova
  \onecopy{1}{
    \disciplina{PCS3111}{Laboratório de Programação Orientada a Objetos}
    \avaliacao{1a. Prova}
    \data{31/08/2016}
    \turmas{1/2M, 2/2T, 3/3M, 4/4M, 5/4T, 6/5M, 7/5T, 8/6T}
    \cabecalho{}
    
    \centerline{\textbf{INSTRUÇÕES}}

\begin{enumerate}
	\item Esta prova contém 2 (duas) questões dissertativas e 8 (oito) testes. Verifique atentamente se todas elas estão presentes. Em caso de dúvida, chame o professor.
	\item \emph{Preencha o seu nome, a sua turma e o seu número USP no início da prova}. Provas não identificadas não serão corrigidas e poderão ser recolhidas durante a sua execução pelo professor.\textbf{ Caso o seu número USP possua 7 dígitos, inicie o preenchimento com ZERO, da esquerda para direita.}
	\item Resolva as questões dissertativas apenas nas folhas de questão. Respostas fornecidas fora do local a elas destinadas não serão corrigidas.
	\item A duração total da prova é de 100 minutos.
	\item É proibido o uso de calculadoras, computadores, celulares e outros equipamentos eletrônicos.
	\item É proibido retirar o grampo da prova.
	\item A interpretação das questões faz parte da avaliação dos alunos. Caso considere alguma hipótese que não esteja explicitada no enunciado, indique claramente na resposta.
	\item A prova é \textbf{SEM CONSULTA}.
\end{enumerate}

Boa sorte!\\



	\noindent{}{\large{}\textbf{[5,2 pontos] Parte I - Questões Dissertativas}}\par
	\vspace{1cm}

    \insertgroup{dissertativas}
	  % \noindent{}{\large{}\textbf{[4,8 pontos] Testes}}\par
    % \def\AMCbeginQuestion#1#2{\par\noindent{\bf Teste #1} #2\hspace*{1em}}
    % Insere as questões aleatorizadas, começando pelo número abaixo
    % \AMCnumero{1}
    
	\newpage
	\noindent{}{\large{}\textbf{[4,8 pontos] Parte II - Testes}}\par
	\vspace{1cm}

    \shufflegroup{testes}
    \insertgroup{testes}
  }
\end{document}